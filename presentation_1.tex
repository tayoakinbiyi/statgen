%%%%%%%%%%%%%%%%%%%%%%%%%%%%%%%%%%%%%%%%%
% Beamer Presentation
% LaTeX Template
% Version 1.0 (10/11/12)
%
% This template has been downloaded from:
% http://www.LaTeXTemplates.com
%
% License:
% CC BY-NC-SA 3.0 (http://creativecommons.org/licenses/by-nc-sa/3.0/)
%
%%%%%%%%%%%%%%%%%%%%%%%%%%%%%%%%%%%%%%%%%

%----------------------------------------------------------------------------------------
%	PACKAGES AND THEMES
%----------------------------------------------------------------------------------------

\documentclass{beamer}

\mode<presentation> {

% The Beamer class comes with a number of default slide themes
% which change the colors and layouts of slides. Below this is a list
% of all the themes, uncomment each in turn to see what they look like.

%\usetheme{default}
%\usetheme{AnnArbor}
%\usetheme{Antibes}
%\usetheme{Bergen}
%\usetheme{Berkeley}
%\usetheme{Berlin}
%\usetheme{Boadilla}
%\usetheme{CambridgeUS}
%\usetheme{Copenhagen}
%\usetheme{Darmstadt}
%\usetheme{Dresden}
%\usetheme{Frankfurt}
%\usetheme{Goettingen}
%\usetheme{Hannover}
%\usetheme{Ilmenau}
%\usetheme{JuanLesPins}
%\usetheme{Luebeck}
\usetheme{Madrid}
%\usetheme{Malmoe}
%\usetheme{Marburg}
%\usetheme{Montpellier}
%\usetheme{PaloAlto}
%\usetheme{Pittsburgh}
%\usetheme{Rochester}
%\usetheme{Singapore}
%\usetheme{Szeged}
%\usetheme{Warsaw}
\usepackage{comment}

% As well as themes, the Beamer class has a number of color themes
% for any slide theme. Uncomment each of these in turn to see how it
% changes the colors of your current slide theme.

%\usecolortheme{albatross}
%\usecolortheme{beaver}
%\usecolortheme{beetle}
%\usecolortheme{crane}
%\usecolortheme{dolphin}
%\usecolortheme{dove}
%\usecolortheme{fly}
%\usecolortheme{lily}
%\usecolortheme{orchid}
%\usecolortheme{rose}
%\usecolortheme{seagull}
%\usecolortheme{seahorse}
%\usecolortheme{whale}
%\usecolortheme{wolverine}

%\setbeamertemplate{footline} % To remove the footer line in all slides uncomment this line
%\setbeamertemplate{footline}[page number] % To replace the footer line in all slides with a simple slide count uncomment this line

%\setbeamertemplate{navigation symbols}{} % To remove the navigation symbols from the bottom of all slides uncomment this line
}

\usepackage{graphicx} % Allows including images
\usepackage{booktabs} % Allows the use of \toprule, \midrule and \bottomrule in tables

%----------------------------------------------------------------------------------------
%	TITLE PAGE
%----------------------------------------------------------------------------------------

\title[local levels]{Methods for combining multiple sub-significant association signals, with application to trans eQTL detection and detection of epistasis in complex trait GWAS} 

\author{Takintayo Akinbiyi} 
\institute[U of C] 
{
University of Chicago \\ % Your institution for the title page
\medskip
\textit{akinbiyi2000@gmail.com} % Your email address
}
\date{\today} % Date, can be changed to a custom date

\begin{document}

\begin{frame}
\titlepage
 
\end{frame}

\begin{frame}
\frametitle{Overview} 
\tableofcontents 
\end{frame}

\begin{frame}
\begin{enumerate}
\item 
\item 
\item Variance component approach(es) (they differ a bit between the 2 problems)
\item Other possible approaches
\item Local levels approach
\item Implementation of local levels in IID case
\item Incorporation of dependence into local levels approach (and also into variance component approach to the extent that it is not already discussed under "variance component approach")
\item Application to trans eQTL detection
\item Application to epistasis detection
\end{enumerate}
\end{frame}

\section{Brief overview of trans eQTL detection problem}

\begin{frame}
\frametitle{Overview of problem}
\begin{itemize}
\item We will develop novel and powerful methods for discovering trans eQTLs and epistatic SNP pairs (I would not use this.  It's sort of "grant proposal speak", i.e., selling the idea.)
\item There is a significant multiple testing problem as there are a huge number of SNP-transcript combinations as well as potential SNP x SNP candidates for epistasis
\item Correlation between related transcripts and between proximate SNPs will induce correlation between individual transcript x SNP test stats and between individual SNP x SNP test stats
\item True epistatic SNP pairs and trans eQTLs may be rare and present weak signals
\item When there are so many more true NULLs than not, the largest signals will come from coincidentally large false signals
\end{itemize}
\end{frame}

\begin{frame}
\frametitle{Some similarities with problem of detection of epistasis in complex trait mapping}
\begin{itemize}
\item As we will be trying to incorporate correlation between signals into our methodology, we note some differences between the two problem domains
\item For epistasis, estimation of correlation between test stats for individual SNP x SNP pairs will likely require the raw data whereas summary data (Z scores for individual transcript x SNP pairs) may be sufficient for trans eQTLs
\item Correlation structure in epistasis may have a simpler more exchangeable structure as it is dependent mostly on proximity and LD
\end{itemize}
\end{frame}

\section{Consideration of most significant signals only vs.\ combining multiple possibly sub-significant signals}

\begin{frame}
\frametitle{Single Global Null}
\begin{itemize}
\item Switch from many single gene x gene or snp x transcript test to single global tests of a snp being a trans eQTL for one or more transcript or a snp being in epistasis with at least one other snp for a particular phenotype
\item GWAS approach of looking at which signals cross genome wide threshold leaves out sub-significant signals that are relevant but not hugely strong
\item Can’t identify these individually but can maybe say something about them as a group like they appear often with certain type of snps
\item Genome wide threshold is both conservative due to not incorporating between test correlation and low power because the threshold has to be so high to maintain correct global size
\end{itemize}
\end{frame}

% dive into
\begin{frame}
\frametitle{Trans eQTL Notation}
Our model for observing $T$ transcript levels each from $N$ individuals is as follows
$$Y=X\beta +G\alpha +u+\varepsilon$$
\begin{itemize}
\item $Y: N\times T$ is the matrix of transcript levels
\item $X: N\times c$ is the matrix of $c$ covariates
\item $\beta: c\times T$ is the matrix of coefficients
\item $G: N\times 1$ is a vector of values the SNP takes
\item $\alpha: 1\times T$ is the vector of fixed effect of the SNP
\item $u: N\times T \sim N(0,K\otimes V_1)$ is the genotype matrix
\item $\epsilon: N\times T\sim N(0,I\otimes V_2)$ is the error matrix
\item $K$ is the GRM
\end{itemize}
\end{frame}

\begin{frame}
\frametitle{notation}
\begin{itemize}
\item Our global Null hypotheses is that the SNP is not a trans eQTL for any gene: $\forall t,\ \alpha_t=0$
\item Let $Z_t=\hat{\alpha}/s.e.(\hat{\alpha})\sim N(0,1)$ under the local null hypothesis that the SNP is not a trans eQTL for transcript $t$: $\alpha_t=0$
\item The p-value of $Z_t$ under the local null is denoted $\pi_t$
\item We will allow for a correlation between the $Z_t$ such that $Z=[Z_1,...,Z_n]^T\sim MVN(0,\Sigma)$ under the global null
\item To begin we consider the following alternative hypothesis:
$$Z_i\sim \epsilon N(\mu,1) + (1-\epsilon)N(0,1),\ cor(Z_i,Z_j)=\Sigma_{ij}$$
\item Typically, $T^{-1}\le\epsilon\le T^{-1/2}$ and $\mu$ will be "weak"
\end{itemize}
\end{frame}

\section{Variance component approach(es) (they differ a bit between the 2 problems) }

\begin{frame}
\frametitle{Predominant Existing Method}
non-local levels methods:
Variance components methods: Crawford et al 2017 for epistasis find ref using for eqtl 
sketch out this approach as it’s main approach in gwas
what are strengths/weaknesses
look at “monster” paper by mary sara, xihong lin “skat” paper see basic idea
focus on trans eQTL i.e. epistasis is tricky will talk about it at end
\end{frame}

\section{Other possible approaches}

\begin{frame}
\frametitle{Alternatives to Equal Local Levels}
\begin{itemize}
\item Higher criticism.  Turns out to work well in finite samples and has asymptotic optimality in some settings but asymptotic accuracy poor until very large $T$
$$HC=\max_{1\le t\le T/2\ :\ \pi_{(t)}>1/T}\sqrt{T}\frac{t/T-\pi_{(t)}}{\sqrt{\pi_{(t)}(1-\pi_{(t)})}}$$
\item FDR. Underpowered because it implicitly assumes alternatives have strong signals and will thus be among the smallest p-values
$$FDR=I\left\{\exists\ t\ \ s.t.\ \ \frac{\pi_{(t)}}{t/T}\le 0.05\right\}$$
\item Berk Jones. An asymptotic approximation to equal local levels
$$BJ=\max_{1\le t\le T/2}\pi_{(t)}\log\left(\frac{\pi_{(t)}}{t/T}\right)+(1-\pi_{(t)})\log\left(\frac{1-\pi_{(t)}}{1-t/T}\right)$$
\end{itemize}
\end{frame}

\section{Local Levels Approach}

% dive into last bullet
\begin{frame}
\frametitle{Local Levels}
\begin{itemize}
\item Our approach will set a critical value $h_t$ for each $\pi_{(t)}$ and reject the global null if $\pi_{(t)}\le h_t$ for any $t$
\item In the literature $\eta_t=P_0\left[\pi_{(t)}\le h_t\right]$ is referred to as the local level
\item This leads to a significance level $\alpha$ for testing the global null: $\alpha=1-P_0\left[\forall t:\ \pi_{(t)}>h_t\right]$
\item We will choose to set $\forall t,\ \eta_t=\eta$ known as equal local levels, an approach first proposed by Berk \& Jones 1979 but subsequently under-utilized
\item The motivation for equal local levels was given in Berk \& Jones 1979 who showed they were efficient in a particular sense
\end{itemize}
\end{frame}

\section{Implementation of local levels in IID case}

\begin{frame}
\frametitle{Deriving A Candidate Statistic}
\begin{itemize}
\item Suppose $\Sigma=I$, then under the null, $\pi_{(t)}\sim\ Beta(t,\ T+1-t)$, and we can easily calculate the $h_t$ for any given $\eta$
\item If $\Sigma\neq I$, then we can extend the above
\item Let $S(t)=\sum_k I\left\{|Z_k|\ge t\right\}$ where $I$ is the indicator and let $\Phi$ be the CDF of N(0,1).   
\end{itemize}
\begin{eqnarray*}
P_0\left[\pi_{ (t) }\le h_t\right]&=&P_0\left[|Z|_{(T+1-t)}\ge -\Phi^{-1}(h_t/2)\right]\\
&=&P_0\left[\left(\sum_{ k } I\left\{|Z_k| \ge -\Phi^{-1}(h_t/2)\right\}\right) \ge i\right] \\
&=& P_0\left[S\left(-\Phi^{-1}(h_t/2)\right)\ge t\right]
\end{eqnarray*}
\end{frame}

\subsection{Incorporation of dependence into local levels approach}

\begin{frame}
\frametitle{Deriving a candidate Test Statistic}
\begin{itemize}
\item When $\Sigma\ne I$, $S(t)$ has the null distribution of an over or under-dispersed binomial.  
\item A possible method is to approximate the distribution of $S(t)$ with the Extended Beta Binomial distribution $EBB(\lambda,\gamma)$ where we choose $\lambda$ and $\gamma$ to match the mean and variance of $S(t)$.  Let $f(x;t)$ be its PMF.
\item We choose $h_t$ such that 
\begin{eqnarray*}
\eta&=&P_0\left[S\left(-\Phi^{-1}\left(h_t/2)\right)\right)\ge t\right]\\
&\approx&1-\sum_{k=0}^{t-1}f\left(k\ ;\ -\Phi^{-1}\left(h_t/2\right)\right)
\end{eqnarray*}
\end{itemize}
\end{frame}

\begin{frame}
\frametitle{Deriving a Candidate Test Statistic}
\begin{itemize}
\item $P(S(t)\ge x)$ should be decreasing in $t$ for any $x$.  And $-\Phi^{-1}(x/2)$ should be decreasing in $x$.  Thus in practice given an $\eta$ we can reject the global null if for any $t$
\begin{eqnarray*}
\eta&\ge& 1-\sum_{k=0}^{t-1}f\left(k\ ;\ -\Phi^{-1}(\pi_{(t)}/2)\right)\\
&=& 1-\sum_{k=0}^{t-1}f\left(k\ ;\ |Z|_{(T+1-t)}\right)
\end{eqnarray*}
\item Our statistic is then
\end{itemize}
$$LL=\min_{1\le t\le T/2}\left(1-\sum_{k=0}^{t-1}f\left(k\ ;\ |Z|_{(T+1-t)}\right)\right)$$
\end{frame}

\begin{frame}
\frametitle{Calculating $\alpha$}
\begin{itemize}
\item We need to choose $\eta$ so that we end up with the correct $\alpha$ and we have some candidate reasonable assumptions that can be employed
\item \textbf{Monte Carlo} We repeatedly sample $Z\sim MVN(0,\Sigma)$ calculate $LL$ for each one and then find the $\alpha$ quantile of sample of $LL$
\item \textbf{Recursive}  Let $b_t=-\Phi^{-1}\left(h_t/2\right)$
\begin{eqnarray*}
PP(LL<\eta)&=&P\left(\forall t,\ |Z|_{(T+1-t)}<b_t\right)\\
&=&P\left(\forall t,\ S\left(b_t\right)\le t-1\right)\\
&=&q_{t,0}
\end{eqnarray*}
\end{itemize}
\end{frame}

\begin{frame}
\frametitle{Calculating $\alpha$}
\begin{eqnarray*}
q_{t,a}&=&P\left(S\left(b_t\right)=a,\ \bigcap_{k=1}^{t-1}S\left(b_k\right)\le t-k\right)\\
&=&\sum_{m=a}^{T-t+1}P\left(S\left(b_t\right)=a\left\|\ S\left(b_{t-1}\right)=m,\ \bigcap_{k=1}^{t-2}S\left(b_k\right)\le t-k\right.\right)q_{t-1,m}
\end{eqnarray*}
We can approximate this last quantity by assuming $S(b_t)$ is Markov in $t$
$$q_{t,a}\approx\sum_{m=a}^{T-t+1}P\left(S\left(b_t\right)=a\left\|\ S\left(b_{t-1}\right)=m\right.\right)q_{t-1,m}$$
\end{frame}

\begin{frame}
\frametitle{$q_{t,a}$}
\begin{itemize}
\item We will explore approximations to the conditional distribution of $S(b_t)$
\item One option is to approximate it with an EBB distribution matching mean and variance
\item Part of our work will be to evaluate these assumptions and explore alternatives
\end{itemize}
\end{frame}

\begin{frame}
\frametitle{Estimate $\Sigma$}

\section{Application to trans eQTL detection}

\section{Application to epistasis detection}

\end{frame}


%------------------------------------------------

\begin{comment}
\begin{frame}
\frametitle{References}
\footnotesize{
\begin{thebibliography}{99} % Beamer does not support BibTeX so references must be inserted manually as below
\bibitem[Smith, 2012]{p1} John Smith (2012)
\newblock Title of the publication
\newblock \emph{Journal Name} 12(3), 45 -- 678.
\end{thebibliography}
}
\end{frame}
\end{comment}

%------------------------------------------------

\begin{frame}
\Huge{\centerline{The End}}
\end{frame}

%----------------------------------------------------------------------------------------

\end{document} 